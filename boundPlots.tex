\documentclass[12pt,a4paper]{article}
\usepackage[italian]{babel}
\usepackage[left=2.5cm,top=3cm,right=2.5cm,bottom=3cm]{geometry}
\usepackage{subcaption}
\usepackage{tikz}
\usepackage{pgfplots,pgfplotstable}
\usepackage{booktabs}
\usepackage{amsmath}

\pagenumbering{gobble}
\pgfplotstableset
{
    every head row/.style={before row=\toprule,after row=\midrule},
    every last row/.style={after row=\bottomrule},
}

\begin{document}

Potenziale usato:
\begin{align*}
    V(r) &= V_0\,e^{-\left(r/R_0\right)^2} & V_0 &=-60.575\ \mathrm{MeV},\ R_0 = 1.65\ \mathrm{fm}
\end{align*}

\begin{figure}[!h]
    \centering
    \begin{subfigure}{0.6\textwidth}
        \centering
        \begin{tikzpicture}
            \begin{semilogxaxis}
                [
                    xmin = 50, xmax = 5000,
                    %ymin =, ymax=,
                    title = Variazione dell'energia vs Numero di intervalli,
                    title style={xshift=1.5em, yshift=.5em},
                    xlabel = N,
                    ylabel = {$E - E_b\ [\mathrm{MeV}]$},
                    height = .9\textwidth,
                    width = .9\textwidth,
                ]
                \addplot+
                [
                    only marks,
                    %mark = square,
                    color = blue,
                ] table
                [
                    x index = 0,
                    y expr = \thisrowno{1}-2.2255058,
                    header = false,
                ] {boundInter.dat};
            \end{semilogxaxis}
        \end{tikzpicture}
    \end{subfigure}%
    \begin{subfigure}{0.4\textwidth}
        \centering
        \begin{align*}
            E_b &=2.2255058\ \mathrm{MeV}\\
            R_1 &=5\ \mathrm{fm}\\
            R_2 &=40\ \mathrm{fm}
        \end{align*}

        \pgfplotstabletypeset
        [
            header = false,
            columns/0/.style={column name=$N$},
            columns/1/.style=
            {
                column name = $E-E_b\ [\mathrm{MeV}]$,
                preproc/expr={##1-2.2255058},
                precision = 1, sci zerofill = true,
            }
        ] {boundInter.dat}
    \end{subfigure}
    \caption{Sono troncate le cifre successive all'ottava significativa}
\end{figure}

\begin{figure}
    \centering
    \begin{subfigure}{0.6\textwidth}
        \centering
        \begin{tikzpicture}
            \begin{axis}
                [
                    %xmin = , xmax = ,
                    %ymin =, ymax=,
                    title = Variazione dell'energia vs Punto finale,
                    title style={xshift=1.5em, yshift=.5em},
                    xlabel = {$R_2\ [\mathrm{fm}]$},
                    ylabel = {$E - E_b\ [\mathrm{MeV}]$},
                    height = .9\textwidth,
                    width = .9\textwidth,
                ]
                \addplot+
                [
                    only marks,
                    %mark = square,
                    color = blue,
                ] table
                [
                    x index = 0,
                    y expr = \thisrowno{1}-2.2255058,
                    header = false,
                ] {boundTail.dat};
            \end{axis}
        \end{tikzpicture}
    \end{subfigure}%
    \begin{subfigure}{0.4\textwidth}
        \centering
        \begin{align*}
            E_b &=2.2255058\ \mathrm{MeV}\\
            R_1 &=5\ \mathrm{fm}\\
            N &=4000
        \end{align*}

        \pgfplotstabletypeset
        [
            header = false,
            columns/0/.style={column name=$R_2\ [\mathrm{fm}]$},
            columns/1/.style=
            {
                column name = $E-E_b\ [\mathrm{MeV}]$,
                preproc/expr={##1-2.2255058},
                precision = 1, sci zerofill = true,
            }
        ] {boundTail.dat}
    \end{subfigure}
\end{figure}

\begin{figure}
    \centering
    \begin{subfigure}{0.6\textwidth}
        \centering
        \begin{tikzpicture}
            \begin{axis}
                [
                    %xmin = , xmax = ,
                    %ymin =, ymax=,
                    title = Variazione dell'energia vs Punto di matching,
                    title style={xshift=1.5em, yshift=.5em},
                    xlabel = {$R_1\ [\mathrm{fm}]$},
                    ylabel = {$E - E_b\ [\mathrm{MeV}]$},
                    height = .9\textwidth,
                    width = .9\textwidth,
                ]
                \addplot+
                [
                    only marks,
                    %mark = square,
                    color = blue,
                ] table
                [
                    x index = 0,
                    y expr = \thisrowno{1}-2.2255058,
                    header = false,
                ] {boundMatch.dat};
            \end{axis}
        \end{tikzpicture}
    \end{subfigure}%
    \begin{subfigure}{0.4\textwidth}
        \centering
        \begin{align*}
            E_b &=2.2255058\ \mathrm{MeV}\\
            R_2 &=50\ \mathrm{fm}\\
            N &=4000
        \end{align*}

        \pgfplotstabletypeset
        [
            header = false,
            columns/0/.style=
            {
                column name = {$R_1\ [\mathrm{fm}]$},
                precision = 1, zerofill = true,
            },
            columns/1/.style=
            {
                column name = $E-E_b\ [\mathrm{MeV}]$,
                preproc/expr={##1-2.2255058},
                precision = 1, sci zerofill = true,
            },
        ] {boundMatch.dat}
    \end{subfigure}
\end{figure}

\end{document}
